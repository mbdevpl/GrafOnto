% GrafOnto: interactive graphical visualisation of categorial ontologies
% Copyright (C) 2012  Mateusz Bysiek, http://mbdev.pl/
% 
% This program is free software: you can redistribute it and/or modify
% it under the terms of the GNU General Public License as published by
% the Free Software Foundation, either version 3 of the License, or
% (at your option) any later version.
% 
% This program is distributed in the hope that it will be useful,
% but WITHOUT ANY WARRANTY; without even the implied warranty of
% MERCHANTABILITY or FITNESS FOR A PARTICULAR PURPOSE.  See the
% GNU General Public License for more details.
% 
% You should have received a copy of the GNU General Public License
% along with this program.  If not, see <http://www.gnu.org/licenses/>.

\documentclass{article}
\usepackage[cm]{fullpage} %very small margins (around 1.5cm)
\usepackage{minted} %syntax coloring \begin{minted}{language}

\begin{document}

\title{GrafOnto user guide}
\date{September 11, 2012}
\author{Mateusz Bysiek}
\maketitle

\tableofcontents


\pagebreak[4]


\section{Introduction}

\subsection{License}
GNU General Public License as published by the Free Software Foundation, either version 3 of the
License, or (at your option) any later version.

\subsection{Qt framework}
This application is written with help of Qt framework. It was compiled for Qt 4.8.1.

\subsection{Two applications}
GrafOnto is split into two separate applications: \textit{GrafOntoGui} (from now on called
\textit{Gui}) and \textit{GrafOntoConsole} (from now on called \textit{Console}).
The first one runs in window, and the second in terminal/command-line.

However, the difference between them is only in the presentation layer i.e. the engine that performs
operations on the ontology is the same.

\subsection{Language}
Operation of the application is based on text commands given in language similar to SQL. 
From now on, let's call it \textit{Lang}. It is very simple, below is the basic example:
\begin{minted}{ruby}
add category (thing, situation) # define two categories, called thing and situation
add thing (writing,book,dictionary,title,index) # create five new things 
add situation (is , has) # create two new situations: 'is' and 'has'
dictionary is : book # create a link: (dictionary,is)->(book)
book is: writing
book has: title
dictionary has: index
set situation relation # indicate that category called 'situation' is a relation
# this will cause statements like (dictionary,is)->(book) be seen 
# more like (dictionary)--is-->(book) by the application
set is transitive # indicate that situation 'is' applied to a chain of elements
# like in our example (dictonary,is)->(book); (book,is)-->(writing) 
# or more simply: dictonary--is-->book; book--is-->writing
# will be treated like: dictonary--is-->book--is-->writing by the application
find all that is: writing # should find book and dictionary
\end{minted}

This is the introduction, therefore I will not provide a list of all possible commands right now.
Full specification of Lang is provided later on. Inconsistency in whitespace characters is
deliberate, it is to show that they do not matter and the script will work just as well no matter
how many spaces you will insert between the terms.

\section{Known issues}
Yet unsolved issues.

\begin{itemize}

  \item Space after colon is mandatory.

  \item transitivity does not ``transfer'' other connections of the given element

  \item ``find all that \ldots'' does not always work properly with transitivity

  \item -auto command-line option performs well only when there is only one further argument -
  please use quotes!

\end{itemize}

\newpage

\section{Feature checklist}
This project still remains in development phase, this list provides overview of features that are
really present.

\subsection{Implemented features}
Features listed below are fully implemented:

\begin{itemize}

  \item add category/element (single and multiple)

  \item find all

\end{itemize}

\subsection{Unimplemented features}
Features listed below are not implemented at all or in a degree that is too low to do anything
useful.

\begin{itemize}

  \item deleting

  \item Saving/loading ontologies from files

\end{itemize}

\section{Script language specification}

\subsection{Ontology modification}

\subsubsection{add}
Used to enlarge set of categories and elements.

\begin{minted}{ruby}
add category <category_name>
\end{minted}
Unless category with a given name exists, it is defined. Category cannot 
have name ``category''.

\begin{minted}{ruby}
add <category_name> <element_name>
\end{minted}
If a category exists, and element of a given name does not, new element is created.

\subsubsection{set}
Used to modify category/element properties. List of properties of categories 
is different than list of properties of elements. 

\begin{minted}{c}
set <category_name> relation
set <category_name> not relation
\end{minted}
Sets a given category to be (or not to be) a relation. If the category already is 
(or already is not) a relation, this command does nothing.

\begin{minted}{c}
set <element_name> transitive
set <element_name> not transitive
\end{minted}
Sets a given element to be (or not to be) transitive. If the element already is 
(or already is not) transitive, this command does nothing.

\newpage

\subsection{Ontology persistence}

\subsubsection{save}
\begin{minted}{ruby}
save <filename>
\end{minted}
Saves the current ontology to a XML file. Format of the file is described later on.

\subsubsection{load}
\begin{minted}{c}
load <filename>
\end{minted}
Loads the ontology from a XML file, completely erasing and the current ontology.

\subsection{Ontology querying}
\subsubsection{find}
\begin{minted}{ruby}
find all
\end{minted}
Displays all categories, elements, cells and statements currently present in the ontology.

\begin{minted}{ruby}
find all that <partial_statement>
\end{minted}
Searches for all matching statements in the ontology. Matching statement is defined as
a statement that has right hand side identical to the given one, and left hand side being 
a superset of a given LHS.

\newpage

\section{Command-line options and arguments}
GrafOnto supports the following arguments:

\verb|-demo| - Supported in both Gui and Console, causes application to load an example ontology at
start-up, before execution of any other operations.

\verb|-auto| - Supported only in Console, causes application to interpret all further arguments 
as commands in Lang. Since commands are separated by spaces, and spaces are a must in all Lang commands,
all further commands are first joined into one long text, and then are split where semicolons (\verb|;|).

\verb|-debug| - Supported only by debug builds. Causes debug output to be displayed. Information 
about what build are you running can be 
displayed in Gui via \textit{About} menu option, and in Console it is shown at start of the program.

\section{XML ontology container specification}
This is a specification of XML files that are read/written via \verb|load|/\verb|save| commands.

\begin{minted}[linenos,numbersep=5pt]{xml}
<ontology>
	<categories>
		<!-- each category has unique id -->
		<category id="#">name</category>
		<category id="#">name2</category>
		...
	</categories>
	<elements>
		<!-- each element has unique id, and is given a category id; 
		     it also has transitivity (1) or it does not have it (0) -->
		<element id="#" cat_id="#" transitive="0">name</element>
		<element id="#" cat_id="#" transitive="0">name2</element>
		<element id="#" cat_id="#" transitive="1">name3</element>
		...
	</elements>
	<cells>
		<!-- each cell has unique id, and contains a list 
		     of elements (in abbreviated form) -->
		<cell id="#">
			<elem id="#" />
			<elem id="#" />
			<elem id="#" />
			...
		</cell>
		<cell id="#">
			<elem id="#" />
			<elem id="#" />
			...
		</cell>
		...
	</cells>
	<statements>
		<!-- each statement has unique id, and is given an id of cell 
		     on the left side, and id of a cell on the right side -->
		<statement id="#" left="#" right="#" />
	</statements>
</ontology>
\end{minted}

\end{document}
